\documentclass[unicode,master]{scutthesis} % 草稿封面,硕士则添加选项master,博士则去掉。使用正式封面时注释该行
%\documentclass[unicode,master,pdfcover]{scutthesis} %   % 论文正式封面,pdfcover为可选项,终稿再添加,使用草稿封面时注释该行
\usepackage{fontspec,color,array,longtable,graphicx} 
\usepackage{anyfontsize} %消除字体警告
\usepackage{enumitem}
%%%%%%%%%%%%%%%%%%%%%%%%%%%%%%%%%%%%%%%%%%%%%%%%%%%%%%%%%%%%%%%%%%%%%%%%%%%%%%%%%%——by MCH
%编译范围
% \includeonly{chapter04}
%参考文献设置
\usepackage[backend=biber,style=gb7714-2015,gbalign=gb7714-2015,gbpub=false,gbnamefmt = lowercase]{biblatex}
\addbibresource[location=local]{MyLibrary.bib} % 如果在其他盘,改为相对路径。比如F盘,改为:F/MyLibrary.bib
\addbibresource[location=local]{mybibfile2.bib} % 无论什么来源的bib文件,只要由参考文献的BibTeX组成,都可以使用此模板。参考文献的BibTeX获取方法可百度
%页眉页脚设置
\usepackage{fancyhdr}
\usepackage{listings}
\usepackage{xunicode}
\renewcommand{\lstlistingname}{列表}
\pagestyle{fancy}
\fancyfoot[C]{\headfont\thepage}
\renewcommand{\chaptermark}[1]{\markboth{\chaptername\ #1}{}}
\renewcommand{\sectionmark}[1]{\markright{\thesection\ #1}}
\fancyhead[RE]{}
\fancyhead[RO]{}
\fancyhead[LE]{}
\fancyhead[LO]{}
\fancyhead[CO]{\headfont{\leftmark}}
\fancyhead[CE]{\headfont{华南理工大学硕士学位论文}}% 
\renewcommand{\headrulewidth}{1.5pt}
\renewcommand{\footrulewidth}{0pt}
%%%%%%%%%%%%%%%%%%%%%%%%%%%%%%%%%%%%%%%%%%%%%%%%%%%%%%%%%%%%%%%%%%%%%%%%%%%%%%%%%%
\usepackage[unicode=true,bookmarks=true,bookmarksnumbered=true,bookmarksopen=false,breaklinks=false,pdfborder={0 0 1},backref=false,colorlinks=true]{hyperref}
\hypersetup{pdftitle={LaTeX模板使用说明},
	pdfauthor={姓名},
	pdfsubject={华南理工大学硕士学位论文},
%%	pdfsubject={华南理工大学博士学位论文},
	pdfkeywords={PDF关键字1;PDF关键字2},
%%		linkcolor=black, anchorcolor=black, citecolor=black, filecolor=black, menucolor=black, urlcolor=black, pdfstartview=FitH}% 黑白,提交版
	linkcolor=blue, anchorcolor=black, citecolor=red, filecolor=magenta, menucolor=red, urlcolor=magenta, pdfstartview=FitH}% 彩色

\makeatletter
%%%%%%%%%%%%%%%%%%%%%%%%%%%%%% LyX specific LaTeX commands.
\providecommand{\LyX}{\texorpdfstring%
	{L\kern-.1667em\lower.25em\hbox{Y}\kern-.125emX\@}
	{LyX}}
%% Because html converters don't know tabularnewline
\providecommand{\tabularnewline}{\\}
\makeatother
\begin{document}
	%%%%%%%%%%%%%草稿封面设置%%%%%%%%%%%%%使用“正式封面”时不需要理会这部分
	\title{LaTeX模板使用说明}	
	\author{作者姓名}	
	\supervisor{指导教师:xxx\ 教授}	
	\institute{华南理工大学}	
	\date{2020年5月20日}
	%%%%%%%%%%%%%%%%%%%%%%%%%%%%%%%%%%%%%
	\maketitle	
	\frontmatter	%此后为罗马数字页码,页面类型为plain
	\chapter{摘\texorpdfstring{\quad}{}要}
	本模板由Shun Xu\cite{_}以及yecfly\cite{_a}的模板修改而来,适合于华南理工大学硕/博士毕业论文。既然已经入坑LaTeX,就不推荐使用LYX,但本模板在修改祖传代码过程中仅对修改部分进行更新,其余部分仍保留源代码。另外参考文献管理软件推荐使用zotero,这也是本模板使用的软件。本模板最主要的改动是参考文献使用biber,而不是原来的bibtex,因此不再需要.bst文件。

\keywordsCN{\LaTeX{};论文}

\chapter{Abstract}
	

\keywordsEN{\LaTeX{}; Paper} % 中英文摘要
	%%%%%%%%%%%%%%%%%%%%%%%%%%%%%%%%%%%%%%%%%%%%%%%%
	% 目录、表格目录、插图目录这几个字本身的大纲级别是一级的,即和章名有相同的字号字体。目录表的内容通过titletoc宏包在。cls文件设置了。
	%\cleardoublepage % pdfbookmark可能需要这一条才能正常工作
	\pdfbookmark{\contentsname}{toc} %为目录添加pdf文件书签
	\tableofcontents	%目录
	% \listoffigures	%插图目录(可选)
	% \listoftables	%表格目录(可选)
	
	\begingroup
		\renewcommand*{\addvspace}[1]{}
		\newcommand{\loflabel}{图} 
		\renewcommand{\numberline}[1]{\loflabel~#1\hspace*{1em}}	
		\listoffigures
		
		\newcommand{\lotlabel}{表}
		\renewcommand{\numberline}[1]{\lotlabel~#1\hspace*{1em}}
		\listoftables
	\endgroup

	%%%%%%%%%%%%%%%%%%%%%%%%%%%%%%%%%%%%%%%%%%%%%%%%%
	\chapter{主要符号对照表}
【本节论文规范为可选,如果你的论文没有相关内容那么去除这一节;如果有,则删除这一行注释。】
\begin{table}
	\centering{}%
	\begin{tabular}{l>{\centering}p{0.5cm}l}
	 $ \bm{X}_n\bm{Y}_n\bm{Z}_n $-地理坐标系           &  & ${\bm{X}_b}{\bm{Y}_b}{\bm{Z}_b}$-机体坐标系\tabularnewline
	 $ \psi $-偏航角								   &  & $\theta$-俯仰角\tabularnewline
	 $\varphi$-滚转角  							   &  & $\bm{R}^n_b$、$\bm{R}$-机体系到NED系的旋转矩阵\tabularnewline
	 $\bm{G}$-NED系的重力  							  &  &   $\varphi_0 $-气动面安装角\tabularnewline
	 $ w $-系统的外部扰动								&  &  $T$-系统采样周期\tabularnewline
	 $\bm{F}$-机体系的气动力 						    &  &   $\bm{M}$-机体系的气动力矩\tabularnewline
	 $\rho$-空气密度 								  &  &  $C_{D,x} $、$ C_{D,y} $、$ C_{D,z} $-沿机体轴阻力系数\tabularnewline
	 $A_x $、$ A_y $、$ A_z $-沿机体轴的截面面积 		 &  &  $v$-机身相对于空气的速度分量\tabularnewline 
	 $l_{a}$-机身气动阻力作用点与重心的距离   			  &  &  $V_c$-气体在无穷远处的速度\tabularnewline
	 $T_d$-涵道体升力  								 &  &  $T_p$-风扇升力\tabularnewline
	 $T_a$-总升力 								      &  &  $q_a$-涵道升力分配系数\tabularnewline
	 $ p_U $-桨盘上表面压强 						   &  &  $p_L$-桨盘下表面压强\tabularnewline
	 $V_c+V_i$-桨盘上下表面气体速度 					 &  &  $S$-桨盘面积\tabularnewline
	 $ V_i $-桨盘处气流诱导速度 						  &  &  $ V_{cr} $-理想自转下降速率\tabularnewline
	 $ Q $-风扇扭矩 								 &  &  $ \varpi $-风扇转速\tabularnewline
	 $\mu$-环绕涵道角度变量 						  &  &  $\hat{\bm{i}}$-沿机体系$x$轴方向的单位矢量\tabularnewline 
	 $\hat{\bm{j}}$-沿机体系$y$轴方向的单位矢量  	   &  &  $C_{l, d}(\alpha_d)$-涵道翼型升力曲线\tabularnewline 
	 $C_{d, d}(\alpha_d)$涵道翼型阻力曲线  		      &  &  $c_d$-涵道翼型弦长\tabularnewline 
	 $C_{l_{\alpha}}$-风管翼型升力曲线斜率  			 &  &  $C_{l, \min }$、$ C_{l, \max } $-升力系数极限\tabularnewline 
	 $C_{d, o }$、$C_{d, g }$-拟合阻力曲线经验常数 	&  &  $R$-风扇半径\tabularnewline 
	 $C_{d u c t}$ - 常值比例系数  					&  &  $l_{d}$-重心与涵道气动力作用点的距离\tabularnewline
	 $k_{\delta}$-操纵面气动升力系数 				 &  &  $\alpha_d$-攻角\tabularnewline
	 $ I_{b}$-风扇转动惯量  						   &  &  $ d_{af} $ 、$ d_{ds} $-风扇扭矩常系数\tabularnewline
	 $\bm{L}_{{r}}$-风扇角动量  						&  & \tabularnewline 					
	\end{tabular}
\end{table}	% 符号对照表(可选)
	\chapter{英文缩略词}
【本节论文规范为可选,如果你的论文没有相关内容那么去除这一节;如果有,则删除这一行注释。】
\begin{table}
	\centering{}%
	\begin{tabular}{ccc}
		SCUT  & South China University of Technology & 华南理工大学\tabularnewline
		&  & \tabularnewline
		&  & \tabularnewline
		&  & \tabularnewline
		&  & \tabularnewline
	\end{tabular}
\end{table} 	% 缩略词	
	
	\mainmatter %此后为阿拉伯数字页码
	
    %%%%%%%%%%%%%%%%%%%%%%%%%%%%%%%%%%%%%%%%%%%%%%页眉页脚设置 ——by MCH 
    \fancypagestyle{plain}{
    	\pagestyle{fancy}
    }	% 每章的第一页会默认使用plain,没有页眉。通过重定义plain为fancy解决
    \pagestyle{fancy}	%设置页眉页脚为fancy
    %%%%%%%%%%%%%%%%%%%%%%%%%%%%%%%%%%%%%%%%%%%%%%分章节,结合导言区的\includeonly命令可仅编译部分章节,编译时不用切换界面,直接在相应章节编译即可。
	\chapter{绪论}
%
\section{研究背景和意义}
%

这里主要是想推荐一种“学术生态”,即利用各种工具展开科研工作,以达到事半功倍的效果。需要用到以下软件:
\begin{enumerate}[topsep = 0 pt, itemsep= 0 pt, parsep=0pt, partopsep=0pt, leftmargin=44pt, itemindent=0pt, labelsep=6pt, label=(\arabic*)]
	\item 	参考文献管理软件zotero\cite{_m}。
	\item	可截图获取文献中公式的软件mathpix\cite{_h}。目前开源/免费的替代工具为:。\href{https://www.simpletex.cn/}{SimpleTex}和\href{https://p2t.breezedeus.com/}{Pix2Tex}。目前SimpleTex性能比较好,免费但不开源,不排除未来收费的可能
	\item	TeXlive202x、TeXstudio,相当于开发环境和IDE。本模板是基于TeX的发行版TeXlive202x和编辑器TeXstudio进行的,百度这两个关键字分别安装。关于TeXstudio的使用(快捷键等)可另行查找资料。模板还支持更多ide,更多编译方式见GitHub首页readme.md。若在其他窗口打开了编译生成的pdf文件,记得关掉再编译,否则报错。TeXstudio的设置见第二章。
\end{enumerate}

\section{国内外研究现状}
\section{本文的主要工作和内容安排}

本文的章节安排如下:

第一章,绪论。

第二章,模板简介。主要介绍各文件的内容。

第三章,常用环境。介绍论文写作中常用的环境,包括:图、表、公式、定理。基本涵盖了常用的命令。

%第三章,参考文献设置。本模板对旧版的改动主要是参考文献部分,本章将简单参考文献设置以及
%编译选项的设置等等。


%第一章
	\chapter{ROS系统与多模态特征融合的多任务学习介绍}
%

\section{引言}

将封面打印保存为 thesis\_cover.pdf 文件,硕士使用master\_cover.docx ,博士使用 doctor\_cover.doc 。如果有更新版本的封面,可自行替换。文档类默认是博士论文,下面指令将控制添加封面与否:
\begin{lstlisting}
\documentclass[unicode,master,pdfcover]{scutthesis}	% 使用pdf文件封面的 硕士模板
\documentclass[unicode,master]{scutthesis}	% 不使用pdf文件封面的 硕士模板
\documentclass[unicode,pdfcover]{scutthesis}	% 使用pdf文件封面的博士模板
\documentclass[unicode]{scutthesis}	% 不使用pdf文件封面的博士模板
\end{lstlisting}

\section{ROS机器人操作系统}
\subsection{ROS系统架构}
\subsection{ROS话题通信机制}

\section{机器人仿真模型与各模块介绍}
\subsection{机器人运动模型}
\subsection{自主定位模块}
\subsection{RGBD相机模块}

\section{基于多模态特征融合的多任务学习}
\subsection{多模态学习}
\subsection{多任务学习}
\section{本章小结}


%第二章
	
\chapter{基于多模态特征融合的交通目标检测与可行驶区域分割}

\section{引言}

\section{多模态特征融合模型HybridNets-CLIP}
\subsection{多任务学习模型HybridNets}
\subsection{文字提示模型CLIP}
\subsection{融合文字提示和图像的模型HybridNets-CLIP}

\section{基于多模态特征融合的交通目标检测与可行驶区域分割实验}
\subsection{数据集制作}
\subsection{模型训练}
\subsection{实验结果对比}

\section{本章小结}










%第三章
	\chapter{基于深度强化学习和ROS的机器人导航系统设计}

\section{引言}

\section{基于深度强化学习的机器人导航算法设计}
\subsection{动作空间与状态空间设计}
\subsection{深度强化学习模型设计}
\subsection{深度强化学习模型训练}

\section{本章小结}

	\chapter{机器人路径导航系统测试}
%

\section{引言}
\section{ROS机器人系统通信结构设计}
\section{Gazebo虚拟环境机器人仿真试验}
\section{本章小结}



	% 自行根据需要添加章节。

	\backmatter %章节不编号但页码继续
	%%%%%%%%%%%%%%%%%%%%%%%%%%%%%%%%%%%%%%%%%%%%%%%%%%%%%%%%%%%%%%    微调,使得后续章节的页眉不带章号——by MCH
	\renewcommand{\chaptermark}[1]{\markboth{#1}{}}
	%%%%%%%%%%%%%%%%%%%%%%%%%%%%%%%%%%%%%%%%%%%%%%%%%%%%%%%%%%%%%%
	\chapter{结\texorpdfstring{\quad}{}论}

研究工作总结 

本文不足与展望
 %结论
	 %%%%%%%%%%%%%%%%%%%%%%%%%%%%%%%%%%%%%%%%%%%%%% bibtex参考文献设置  (原版)
%%	\bibliographystyle{scutthesis}
%%	\bibliography{F:/MyLibrary}
	%%%%%%%%%%%%%%%%%%%%%%%%%%%%%%%%%%%%%%%%%%%%%%
	%%%%%%%%%%%%%%%%%%%%%%%%%%%%%%%%%%%%%%%%%%%%%% biber参考文献设置	——by MCH
	%\renewcommand*{\bibfont}{\refbodyfont}			% 设置文献著录字号比正文小一号(五号),需要小四号请注释该行. % 不推荐使用small,而是使用cls文件中精确定义了的字号。
	\phantomsection % “目录”中的链接能正确跳转,需要添加 \phantomsection 否则点击参考文献会跳转到结论
	\addcontentsline{toc}{chapter}{参考文献}	%目录中添加参考文献
	\printbibliography	% 参考文献著录
 	%%%%%%%%%%%%%%%%%%%%%%%%%%%%%%%%%%%%%%%%%%%%%%
 	% 只有一个附录
% 	%%%%%%%%%%%%%%%%%%%此部分为附录环境代码,是比较笨的方法来适应论文撰写规范%%%%%%%%%%%%%%%%%%%%%%%%%%%%%%%%%%%%%%
%对只有一个附录,标题不编号比较美观。
%%%%%%%%%%%%%%%%%%%%%%%%%%%%%%%%%%%%%%%%%%%%%%%%%%%%%%%%%%%%%%%%%%%%%%%%%%%%%%%%%%%%%%%%%%%%%%%%%%%%%%%%%%%%
\setcounter{chapter}{1} %从1开始编号
\setcounter{section}{0}
\setcounter{equation}{0}
\setcounter{table}{0}   
\setcounter{figure}{0}
\chapter{附\texorpdfstring{\quad}{}录} %附录
%%%%%%%%%%%%%%%%%%%%%%%%%%%%%%%%%%%%%%%%%%%%%%%%%%%%%%%%%%%%%%%%%%%%%%%%%%%%%%%%%%%%%%%%%%%%%%%%%%%%%%%%
%%%%%%%%%%%%以下为用户代码,用于撰写您的论文%%%%%%%%%%%%%%%%%%%%%%%%%%%%%%%%%%%%%%%%%%%%%%%%%%%%%%%%%%%%%%


在论文撰写规范中,下面两段话让人费解:

\begin{enumerate}
	\item 	对需要收录于学位论文中但又不适合书写于正文中的附加数据、方案、资料、详细公式推导、计算机程序、统计表、注释等有特色的内容,可做为附录排写,序号采用“附录1”、“附录2”等。	
	\item	公式序号按章编排,如第一章第一个公式序号为“(1-1)”,附录2中的第一个公式为“(2-1)”等。
\end{enumerate}

论文撰写规范要求的附录和通常书籍上使用附录A、附录B等编号的不一样,容易和正文混淆。特殊的要求和代码的耦合,使我不得不使用比较笨的方法来设计附录部分的模板。

\section{测试测试测试}
\subsection{测试测试测试}
%
测试测试测试测试测试测试测试测试测试测试测试测试测试测试测试测试测试测试测试测试测试测试测试测试测试测试测试测试测试测试测试测试测试测试测试测试测试测试测试测试测试测试测试测试测试测试测试测试测试测试测试测试测试测试测试测试测试测试测试测试测试测试测试测试测试测试测试测试测试测试测试测试测试测试测试测试测试测试测试测试测试测试测试测试测试测试测试测试测试测试测试测试测试测试测试测试测试测试测试测试测试测试测试测试测试测试测试测试测试测试测试测试测试测试测试测试测试测试测试测试测试测试测试测试测试测试测试测试测试测试测试测试测试测试测试测试测试测试测试测试测试测试测试测试测试测试测试测试测试测试测试测试测试
\begin{align}
\left\{\begin{array}{l}
\dot{v}_{1}(t)=v_{2}(t) \\
\dot{v}_{2}(t)=R^{2}\left(-\zeta_{1}\left[v_{1}(t)-v_c(t)\right]^{\alpha}-\zeta_{2}\left[\dfrac{v_{2}(t)}{R}\right]^{\beta}\right)
\end{array}\right.	
\end{align}

\begin{align}
\left\{\begin{array}{l}
\dot{v}_{1}(t)=v_{2}(t) \\
\dot{v}_{2}(t)=R^{2}\left(-\zeta_{1}\left[v_{1}(t)-v_c(t)\right]^{\alpha}-\zeta_{2}\left[\dfrac{v_{2}(t)}{R}\right]^{\beta}\right)
\end{array}\right.	
\end{align}
\begin{figure}[htbp]
	\centering	
	\includegraphics[scale=1]{Fig/DFUAV_f31.png}
	\caption{\label{fig_case1}测试测试测试}
\end{figure}
\begin{figure}[htbp]
	\centering	
	\includegraphics[scale=1]{Fig/DFUAV_f31.png}
	\caption{\label{fig_case2}测试测试测试}
\end{figure}
\begin{table}
	\caption{\label{DF_para1}测试测试测试}
	\centering{}%
	\small 
	\begin{tabular}{cccccc}
		\hline 
		参数符号 & 数值&参数符号 & 数值&参数符号 & 数值\tabularnewline
		\hline 
		$ A_x,A_y,A_z $  & $ 0.04082\,\text{m}^2 $ &$ \rho $        &$1.225\,\text{kg}/\text{m}^3$&$ I_b $           & $ 0.000029 $               \tabularnewline
		$ k_{\varpi} $   & $1.13342 \times 10^{-6}$& $ d_{\varpi} $ & $1.13342 \times 10^{-7}$ 	  &$k_{\delta} $     & $ 0.01495 $ 			      \tabularnewline
		$C_{D,x},C_{D,y}$& $ 0.43213 $             &$ C_{D,z} $     & $ 0.13421 $             	  &	$ q_a $ 	     & $ 1.49 $ 				  \tabularnewline
		$ l_{a} $        & $ -0.1121\,\text{m} $   & $ d_{ds} $     & $ 0.01495 $			  	  &$ d_{af} $        & $ 0.01495 $    			  \tabularnewline
		$ R $            & $ 0.11\,\text{m} $      &$ b $           & $ 2 $       			   	  &$ S $ 			 & $ 0.04082\,\text{m}^2 $    \tabularnewline
		$C_{l_{\alpha}}$ & $ 2.212\,/\text{rad} $  &$C_{l, \max } $ & $ 1.05 $ 				   	  &$ C_{l, \min } $  & $ -1.05 $ 				  \tabularnewline
		$ l_2 $          & $ 0.06647\,\text{m} $   &$ l_1 $         & $ 0.17078\,\text{m} $    	  &	$ m $ 		     & $ 1.53\,\text{kg} $ 		  \tabularnewline
		$ C_{d, o } $    & $ 0.9 $                 &$ C_{d, g } $   & $ 0.9 $					  &$ C_{duct} $      & $ 0.78497 $	 			  \tabularnewline
		$ I_x $          & $ 0.02548 $ 			   &$ I_y $         & $ 0.02550 $                 &$ I_z $			 & $ 0.00562 $ 				  \tabularnewline
		\hline 
	\end{tabular}	
\end{table}

\begin{table}
	\caption{\label{TDF_para2}测试测试测试}
	\centering{}%
	\small 
	%	\resizebox{\textwidth}{!}{
	\begin{tabular}{cccccc}
		\hline 
		参数符号 & 数值&参数符号 & 数值&参数符号 & 数值\tabularnewline
		\hline 
		$ I_x $ & $ 054593 $ &$ I_y $ & $ 0.017045 $& $ I_z$ & $ 0.049226 $ \tabularnewline
		$ l_{1} $ & $ 0.0808\,\text{m} $&$ l_{2} $ & $ 0.175\,\text{m} $ &$ l_3 $ & $ 0.06647\,\text{m} $ \tabularnewline 
		$ l_4 $ & $ 0.2415\,\text{m} $ &$ l_5 $ & $ 0.1085\,\text{m} $& $ m $ & $ 3.7\,\text{kg} $ \tabularnewline
		\hline 
	\end{tabular}	%}
\end{table}

\section{测试测试测试}
\subsection{测试测试测试}
%
测试测试测试测试测试测试测试测试测试测试测试测试测试测试测试测试测试测试测试测试测试测试测试测试测试测试测试测试测试测试测试测试测试测试测试测试测试测试测试测试测试测试测试测试测试测试测试测试测试测试测试测试测试测试测试测试测试测试测试测试测试测试测试测试测试测试测试测试测试测试测试测试测试测试测试测试测试测试测试测试测试测试测试测试测试测试测试测试测试测试测试测试测试测试测试测试测试测试测试测试测试测试



 	% 有多个附录
	% %%%%%%%%%%%%%%%%%%%此部分为附录1环境代码,是比较笨的方法来适应论文撰写规范%%%%%%%%%%%%%%%%%%%%%%%%%%%%%%%%%%%%%%
%新增附录时只需要将\setcounter{chapter}{X}以及\chapter{附\texorpdfstring{\quad}{}录 X}中相应的X更改为
%相应的数字,如果只有一个附录,则选用appendix
%%%%%%%%%%%%%%%%%%%%%%%%%%%%%%%%%%%%%%%%%%%%%%%%%%%%%%%%%%%%%%%%%%%%%%%%%%%%%%%%%%%%%%%%%%%%%%%%%%%%%%%%%%%%
\setcounter{chapter}{1} %从1开始编号
\setcounter{section}{0}
\setcounter{equation}{0}
\setcounter{table}{0}   
\setcounter{figure}{0}
\chapter{附\texorpdfstring{\quad}{}录 1} %附录1
%%%%%%%%%%%%%%%%%%%%%%%%%%%%%%%%%%%%%%%%%%%%%%%%%%%%%%%%%%%%%%%%%%%%%%%%%%%%%%%%%%%%%%%%%%%%%%%%%%%%%%%%
%%%%%%%%%%%%以下为用户代码,用于撰写您的论文%%%%%%%%%%%%%%%%%%%%%%%%%%%%%%%%%%%%%%%%%%%%%%%%%%%%%%%%%%%%%%

在论文撰写规范中,下面两段话让人费解:

\begin{enumerate}
	\item 	对需要收录于学位论文中但又不适合书写于正文中的附加数据、方案、资料、详细公式推导、计算机程序、统计表、注释等有特色的内容,可做为附录排写,序号采用“附录1”、“附录2”等。	
	\item	公式序号按章编排,如第一章第一个公式序号为“(1-1)”,附录2中的第一个公式为“(2-1)”等。
\end{enumerate}

论文撰写规范要求的附录和通常书籍上使用附录A、附录B等编号的不一样,容易和正文混淆。特殊的要求和代码的耦合,使我不得不使用比较笨的方法来设计附录部分的模板。

\section{测试一级标题 section}
\subsection{测试二级标题 subsection}
\subsubsection{测试三级标题 subsubsection}
%
测试测试测试测试测试测试测试测试测试测试测试测试测试测试测试测试测试测试测试测试测试测试测试测试测试测试测试测试测试测试测试测试测试测试测试测试测试测试测试测试测试测试测试测试测试测试测试测试测试测试测试测试测试测试测试测试测试测试测试测试测试测试测试测试测试测试测试测试测试测试测试测试测试测试测试测试测试测试测试测试测试测试测试测试测试测试测试测试测试测试测试测试测试测试测试测试测试测试测试测试测试测试测试测试测试测试测试测试测试测试测试测试测试测试测试测试测试测试测试测试测试测试测试测试测试测试测试测试测试测试测试测试测试测试测试测试测试测试测试测试测试测试测试测试测试测试测试测试测试测试测试测试测试
\begin{align}
\left\{\begin{array}{l}
\dot{v}_{1}(t)=v_{2}(t) \\
\dot{v}_{2}(t)=R^{2}\left(-\zeta_{1}\left[v_{1}(t)-v_c(t)\right]^{\alpha}-\zeta_{2}\left[\dfrac{v_{2}(t)}{R}\right]^{\beta}\right)
\end{array}\right.	
\end{align}

\begin{align}
\left\{\begin{array}{l}
\dot{v}_{1}(t)=v_{2}(t) \\
\dot{v}_{2}(t)=R^{2}\left(-\zeta_{1}\left[v_{1}(t)-v_c(t)\right]^{\alpha}-\zeta_{2}\left[\dfrac{v_{2}(t)}{R}\right]^{\beta}\right)
\end{array}\right.	
\end{align}
\begin{figure}[htbp]
	\centering	
	\includegraphics[scale=1]{Fig/DFUAV_f31.png}
	\caption{\label{fig_case1}测试测试测试}
\end{figure}
\begin{figure}[htbp]
	\centering	
	\includegraphics[scale=1]{Fig/DFUAV_f31.png}
	\caption{\label{fig_case2}测试测试测试}
\end{figure}
\begin{table}
	\caption{\label{DF_para1}测试测试测试}
	\centering{}%
	\small 
	\begin{tabular}{cccccc}
		\hline 
		参数符号 & 数值&参数符号 & 数值&参数符号 & 数值\tabularnewline
		\hline 
		$ A_x,A_y,A_z $  & $ 0.04082\,\text{m}^2 $ &$ \rho $        &$1.225\,\text{kg}/\text{m}^3$&$ I_b $           & $ 0.000029 $               \tabularnewline
		$ k_{\varpi} $   & $1.13342 \times 10^{-6}$& $ d_{\varpi} $ & $1.13342 \times 10^{-7}$ 	  &$k_{\delta} $     & $ 0.01495 $ 			      \tabularnewline
		$C_{D,x},C_{D,y}$& $ 0.43213 $             &$ C_{D,z} $     & $ 0.13421 $             	  &	$ q_a $ 	     & $ 1.49 $ 				  \tabularnewline
		$ l_{a} $        & $ -0.1121\,\text{m} $   & $ d_{ds} $     & $ 0.01495 $			  	  &$ d_{af} $        & $ 0.01495 $    			  \tabularnewline
		$ R $            & $ 0.11\,\text{m} $      &$ b $           & $ 2 $       			   	  &$ S $ 			 & $ 0.04082\,\text{m}^2 $    \tabularnewline
		$C_{l_{\alpha}}$ & $ 2.212\,/\text{rad} $  &$C_{l, \max } $ & $ 1.05 $ 				   	  &$ C_{l, \min } $  & $ -1.05 $ 				  \tabularnewline
		$ l_2 $          & $ 0.06647\,\text{m} $   &$ l_1 $         & $ 0.17078\,\text{m} $    	  &	$ m $ 		     & $ 1.53\,\text{kg} $ 		  \tabularnewline
		$ C_{d, o } $    & $ 0.9 $                 &$ C_{d, g } $   & $ 0.9 $					  &$ C_{duct} $      & $ 0.78497 $	 			  \tabularnewline
		$ I_x $          & $ 0.02548 $ 			   &$ I_y $         & $ 0.02550 $                 &$ I_z $			 & $ 0.00562 $ 				  \tabularnewline
		\hline 
	\end{tabular}	
\end{table}

\begin{table}
	\caption{\label{TDF_para2}测试测试测试}
	\centering{}%
	\small 
	%	\resizebox{\textwidth}{!}{
	\begin{tabular}{cccccc}
		\hline 
		参数符号 & 数值&参数符号 & 数值&参数符号 & 数值\tabularnewline
		\hline 
		$ I_x $ & $ 054593 $ &$ I_y $ & $ 0.017045 $& $ I_z$ & $ 0.049226 $ \tabularnewline
		$ l_{1} $ & $ 0.0808\,\text{m} $&$ l_{2} $ & $ 0.175\,\text{m} $ &$ l_3 $ & $ 0.06647\,\text{m} $ \tabularnewline 
		$ l_4 $ & $ 0.2415\,\text{m} $ &$ l_5 $ & $ 0.1085\,\text{m} $& $ m $ & $ 3.7\,\text{kg} $ \tabularnewline
		\hline 
	\end{tabular}	%}
\end{table}

\section{测试测试测试}
\subsection{测试测试测试}
%
测试测试测试测试测试测试测试测试测试测试测试测试测试测试测试测试测试测试测试测试测试测试测试测试测试测试测试测试测试测试测试测试测试测试测试测试测试测试测试测试测试测试测试测试测试测试测试测试测试测试测试测试测试测试测试测试测试测试测试测试测试测试测试测试测试测试测试测试测试测试测试测试测试测试测试测试测试测试测试测试测试测试测试测试测试测试测试测试测试测试测试测试测试测试测试测试测试测试测试测试测试测试

 %附录1
	% %%%%%%%%%%%%%%%%%%%此部分为附录1环境代码,是比较笨的方法来适应论文撰写规范%%%%%%%%%%%%%%%%%%%%%%%%%%%%%%%%%%%%%%
%新增附录时只需要将\setcounter{chapter}{X}以及\chapter{附\texorpdfstring{\quad}{}录 X}中相应的X更改为相应的数字
\setcounter{chapter}{2} %从2开始编号
\setcounter{section}{0}
\setcounter{equation}{0}
\setcounter{table}{0}   
\setcounter{figure}{0}
\chapter{附\texorpdfstring{\quad}{}录 2} %附录2
%%%%%%%%%%%%%%%%%%%%%%%%%%%%%%%%%%%%%%%%%%%%%%%%%%%%%%%%%%%%%%%%%%%%%%%%%%%%%%%%%%%%%%%%%%%%%%%%%%%%%%%%
%%%%%%%%%%%%以下为用户代码,用于撰写您的论文%%%%%%%%%%%%%%%%%%%%%%%%%%%%%%%%%%%%%%%%%%%%%%%%%%%%%%%%%%%%%%

在论文撰写规范中,下面两段话让人费解:

\begin{enumerate}
	\item 	对需要收录于学位论文中但又不适合书写于正文中的附加数据、方案、资料、详细公式推导、计算机程序、统计表、注释等有特色的内容,可做为附录排写,序号采用“附录1”、“附录2”等。	
	\item	公式序号按章编排,如第一章第一个公式序号为“(1-1)”,附录2中的第一个公式为“(2-1)”等。
\end{enumerate}

论文撰写规范要求的附录和通常书籍上使用附录A、附录B等编号的不一样,上述要求最终的效果是这些编号容易和正文的混淆。特殊的要求和代码的耦合,使我不得不使用比较笨的方法来设计附录部分的模板。

\section{测试测试测试}
\subsection{测试测试测试}
%
测试测试测试测试测试测试测试测试测试测试测试测试
\begin{align}
\left\{\begin{array}{l}
\dot{v}_{1}(t)=v_{2}(t) \\
\dot{v}_{2}(t)=R^{2}\left(-\zeta_{1}\left[v_{1}(t)-v_c(t)\right]^{\alpha}-\zeta_{2}\left[\dfrac{v_{2}(t)}{R}\right]^{\beta}\right)
\end{array}\right.	
\end{align}
\begin{align}
\left\{\begin{array}{l}
\dot{v}_{1}(t)=v_{2}(t) \\
\dot{v}_{2}(t)=R^{2}\left(-\zeta_{1}\left[v_{1}(t)-v_c(t)\right]^{\alpha}-\zeta_{2}\left[\dfrac{v_{2}(t)}{R}\right]^{\beta}\right)
\end{array}\right.	
\end{align}
\begin{figure}[htbp]
	\centering	
	\includegraphics[scale=1]{Fig/DFUAV_f31.png}
	\caption{\label{fig_c1}测试测试测试}
\end{figure}
\begin{figure}[htbp]
	\centering	
	\includegraphics[scale=1]{Fig/DFUAV_f31.png}
	\caption{\label{fig_c2}测试测试测试}
\end{figure}
\begin{table}
	\caption{\label{DF_p1}测试测试测试}
	\centering{}%
	\small 
	\begin{tabular}{cccccc}
		\hline 
		参数符号 & 数值&参数符号 & 数值&参数符号 & 数值\tabularnewline
		\hline 
		$ A_x,A_y,A_z $  & $ 0.04082\,\text{m}^2 $ &$ \rho $        &$1.225\,\text{kg}/\text{m}^3$&$ I_b $           & $ 0.000029 $               \tabularnewline
		$ k_{\varpi} $   & $1.13342 \times 10^{-6}$& $ d_{\varpi} $ & $1.13342 \times 10^{-7}$ 	  &$k_{\delta} $     & $ 0.01495 $ 			      \tabularnewline
		$C_{D,x},C_{D,y}$& $ 0.43213 $             &$ C_{D,z} $     & $ 0.13421 $             	  &	$ q_a $ 	     & $ 1.49 $ 				  \tabularnewline
		$ l_{a} $        & $ -0.1121\,\text{m} $   & $ d_{ds} $     & $ 0.01495 $			  	  &$ d_{af} $        & $ 0.01495 $    			  \tabularnewline
		$ R $            & $ 0.11\,\text{m} $      &$ b $           & $ 2 $       			   	  &$ S $ 			 & $ 0.04082\,\text{m}^2 $    \tabularnewline
		$C_{l_{\alpha}}$ & $ 2.212\,/\text{rad} $  &$C_{l, \max } $ & $ 1.05 $ 				   	  &$ C_{l, \min } $  & $ -1.05 $ 				  \tabularnewline
		$ l_2 $          & $ 0.06647\,\text{m} $   &$ l_1 $         & $ 0.17078\,\text{m} $    	  &	$ m $ 		     & $ 1.53\,\text{kg} $ 		  \tabularnewline
		$ C_{d, o } $    & $ 0.9 $                 &$ C_{d, g } $   & $ 0.9 $					  &$ C_{duct} $      & $ 0.78497 $	 			  \tabularnewline
		$ I_x $          & $ 0.02548 $ 			   &$ I_y $         & $ 0.02550 $                 &$ I_z $			 & $ 0.00562 $ 				  \tabularnewline
		\hline 
	\end{tabular}	
\end{table}

\begin{table}
	\caption{\label{TDF_p2}测试测试测试}
	\centering{}%
	\small 
	%	\resizebox{\textwidth}{!}{
	\begin{tabular}{cccccc}
		\hline 
		参数符号 & 数值&参数符号 & 数值&参数符号 & 数值\tabularnewline
		\hline 
		$ I_x $ & $ 054593 $ &$ I_y $ & $ 0.017045 $& $ I_z$ & $ 0.049226 $ \tabularnewline
		$ l_{1} $ & $ 0.0808\,\text{m} $&$ l_{2} $ & $ 0.175\,\text{m} $ &$ l_3 $ & $ 0.06647\,\text{m} $ \tabularnewline 
		$ l_4 $ & $ 0.2415\,\text{m} $ &$ l_5 $ & $ 0.1085\,\text{m} $& $ m $ & $ 3.7\,\text{kg} $ \tabularnewline
		\hline 
	\end{tabular}	%}
\end{table} %附录2
 	%%%%%%%%%%%%%%%%%%%
	\chapter{攻读硕士学位期间取得的研究成果} %博士/硕士记得选其一
\pubfont % 论文撰写规范里,这章是5号宋体,\pubfont 设置字号为5号了。但其实很多论文用小四号也OK。
一、已发表(包括已接受待发表)的论文,以及已投稿、或已成文打算投稿、或拟成文投稿的论文情况\underline{\textbf{(只填写与学位论文内容相关的部分):}}
\begin{table}
	\centering{}%
	\pubfont 
	\begin{longtable}{|>{\centering}m{0.5cm}|m{1.8cm}|>{\centering}m{2.8cm}|>{\centering}m{2.5cm}|>{\centering}m{2.2cm}|>{\centering}m{2.cm}|>{\centering}m{1cm}|}
		\hline 
		\textbf{序号} & \textbf{作者(全体作者,按顺序排列)} & \textbf{题 目} 						   & \textbf{发表或投稿刊物名称、级别} & \textbf{发表的卷期、年月、页码} & \textbf{与学位论文哪一部分(章、节)相关} &\textbf{被索引收录情况}\tabularnewline
		\hline 
		1    & 					  &  &  &  &  &  \tabularnewline
		\hline 
		2	 & 							&  	 &   &  &  & \tabularnewline
		\hline 
	\end{longtable}
\end{table}

注:在“发表的卷期、年月、页码”栏:

1.如果论文已发表,请填写发表的卷期、年月、页码;

2.如果论文已被接受,填写将要发表的卷期、年月;

3.以上都不是,请据实填写“已投稿”,“拟投稿”。

不够请另加页。

二、与学位内容相关的其它成果(包括专利、著作、获奖项目等)



%注:这部分一言难尽,我努力了很久都没有把这个表做好。感觉学校给的这个表的模板非常反人类。看国外大学的博士论文,那种像参考文献著录信息那样一行一行的,比较美观。而这个框框很难放文字进去。

\normalsize % \normalsize可以将下文调回和正文一样的字号,这个随个人喜好。注释掉的话,致谢就就跟随《攻读博士/硕士学位期间取得的研究成果》的字号。 %成果
	\chapter{致\texorpdfstring{\quad}{}谢}


~\\

\begin{minipage}[t]{0.945\textwidth}%
	\begin{flushright}
		作者姓名\\
%		\today\\	% 自动时间
		2020年7月10日\\	%固定时间
		于华南理工大学
		\par\end{flushright}
\end{minipage}

 %致谢
\end{document}
