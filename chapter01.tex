\chapter{绪论}
%
\section{研究背景和意义}
%

这里主要是想推荐一种“学术生态”,即利用各种工具展开科研工作,以达到事半功倍的效果。需要用到以下软件:
\begin{enumerate}[topsep = 0 pt, itemsep= 0 pt, parsep=0pt, partopsep=0pt, leftmargin=44pt, itemindent=0pt, labelsep=6pt, label=(\arabic*)]
	\item 	参考文献管理软件zotero\cite{_m}。
	\item	可截图获取文献中公式的软件mathpix\cite{_h}。目前开源/免费的替代工具为:。\href{https://www.simpletex.cn/}{SimpleTex}和\href{https://p2t.breezedeus.com/}{Pix2Tex}。目前SimpleTex性能比较好,免费但不开源,不排除未来收费的可能
	\item	TeXlive202x、TeXstudio,相当于开发环境和IDE。本模板是基于TeX的发行版TeXlive202x和编辑器TeXstudio进行的,百度这两个关键字分别安装。关于TeXstudio的使用(快捷键等)可另行查找资料。模板还支持更多ide,更多编译方式见GitHub首页readme.md。若在其他窗口打开了编译生成的pdf文件,记得关掉再编译,否则报错。TeXstudio的设置见第二章。
\end{enumerate}

\section{国内外研究现状}
\section{本文的主要工作和内容安排}

本文的章节安排如下:

第一章,绪论。

第二章,模板简介。主要介绍各文件的内容。

第三章,常用环境。介绍论文写作中常用的环境,包括:图、表、公式、定理。基本涵盖了常用的命令。

%第三章,参考文献设置。本模板对旧版的改动主要是参考文献部分,本章将简单参考文献设置以及
%编译选项的设置等等。


